% !TeX root = main.tex
\begin{frame}{}
	\underline{Contribution 4.} Code-Based Fully Dynamic Group Signature (FDGS) and Repudiable-and-Claimable Ring Signature (RCRS) Schemes.
	
	{\small with Khoa Nguyen, Huaxiong Wang, and Neng Zeng}
	
	extended from ASIACRYPT 2019 \cite{NguyenTWZ19}.
\end{frame}

\begin{frame}{FDGS and RCRS Schemes}
	\begin{enumerate}
		\item Group signatures allow a member in a group to anonymously sign on messages on behalf of the group.\pause
		\begin{itemize}
			\item A group is managed by a group manager, which is able to enroll new users and trace identities.\pause
			\item Properties: correctness, anonymity and traceability.\pause
		\end{itemize}
		\item Ring signatures allow any member of an arbitrary set of users can sign on behalf of this set.\pause
		\begin{itemize}
			\item Properties: correctness, anonymity and unforgeability.
		\end{itemize}
	\end{enumerate}
\end{frame}

\begin{frame}{FDGS and RCRS Schemes}
	\underline{Motivation.}\pause
	\begin{itemize}
		\item Standard group signatures do not have any mechanism to revoke users.\pause
		\item Standard ring signatures do not allow users to repudiate or claim the ownership of signatures.\pause
		\item Diversity of assumptions.\pause
	\end{itemize}
	\underline{Results.}\pause
	\begin{itemize}
		\item Code-based fully dynamic group signature scheme.\pause
		\item Repudiable-and-claimable ring signature scheme.\pause
		\item Both have logarithmic size, in the number of users, signatures.
	\end{itemize}
\end{frame}

\begin{frame}{FDGS and RCRS Schemes}
	\underline{Techniques.}\pause
	\begin{enumerate}
		\item Code-based fully dynamic group signature scheme:\pause
		\begin{itemize}
			\item Follow definitional time epoch framework of Bootle et al. \cite{BootleCCGG16} and adapted from Ling et al. \cite{LingNWX17}.\pause
			\item To enroll or revoke users, update paths from Merkle tree and advance to new epoch.\pause
			\item To sign, prove that (i) public key corresponds to signing key, (ii) public key is non-zero and in leaf of the tree, and (iii) correctly encrypts the identity.\pause
		\end{itemize}
		\item Code-based repudiable-and-claimable ring signature scheme:\pause
		\begin{itemize}
			\item User's signing key is $\mathbf{x}$.\pause
			\item To sign, prove that (i) public key corresponds to signing key, (ii) public key is in leaf of the tree, and (iii) $\mathbf{p} = \mathbf{A}\cdot \mathbf{x} \oplus \mathbf{e}$ where $\weight(\mathbf{e}) \leq t$.\pause
			\item User with signing key $\mathbf{x}$ to claim: prove $\mathbf{p} \oplus \mathbf{A}\cdot\mathbf{x} \leq t$. \pause
			\item User with signing key $\mathbf{x}'\not= \mathbf{x}$ to repudiate: prove $\mathbf{p} \oplus\mathbf{A}\cdot\mathbf{x} > t$.
		\end{itemize}
	\end{enumerate}
\end{frame}