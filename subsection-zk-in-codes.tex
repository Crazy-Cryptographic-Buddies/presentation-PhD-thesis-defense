% !TeX root = main.tex
\begin{frame}{ZKAoKs in Code-Based Cryptography}
	In 1993, Stern first proposed an identification protocol \cite{Stern93} based on syndrome decoding problem.\pause
	
	A ZKAoK for secret $\mathbf{x}$ satisfying $\mathbf{A}\cdot\mathbf{x} = \mathbf{y}$ and $\weight(\mathbf{x}) = t$. \pause
	
	Nguyen et al. \cite{NguyenTWZ19, NguyenTWZ19eprint} proposed \pause
	\begin{itemize}
		\item a new commitment scheme based on hardness assumption of $2$-regular null syndrome decoding problem ($\TwoRNSD$).\pause
		\item statistical ZKAoKs for set membership, group and ring signature schemes.\pause 
		\item Soundness error: $(2/3)^\kappa$ where $\kappa$ is the number of repetitions.
	\end{itemize}
\end{frame}

\begin{frame}{ZKAoKs in Code-Based Cryptography}
	
		$\Regular(k,c)$: vector of $k/c$ blocks of $2^c$-bit unit vector.
	
		$\TwoRegular(k,c) = \{\mathbf{x}\oplus\mathbf{y} \in \{0,1\}^{2^c\cdot k / c} : \mathbf{x}, \mathbf{y} \in \Regular(k, c)\}.$
	
		\underline{$\TwoRNSD$ problem.}  Given $n, k, c \in \mathbb{Z}_+$ satisfying $c \mid k$, $\mathbf{A}\uniformly\{0,1\}^{n\times 2^c\cdot k/c}$, the $\TwoRNSD_{n,k,c}$ asks to find $\mathbf{w}$ such that $$\begin{cases}
			\mathbf{A}\cdot \mathbf{w} = \mathbf{0}^{n},\\
			\mathbf{w}\in\TwoRegular(n+ k,c).
		\end{cases}$$
\end{frame}
\begin{frame}{ZKAoKs in Code-Based Cryptography}
	\underline{Abstraction of Stern \cite{NguyenTWZ19}.} \footnote{continue writing here}
\end{frame}

\begin{frame}{ZKAoKs in Code-Based Cryptography}
	\underline{Overview of Results.}\pause
	
	{\small 1. Refined Abstraction of Stern (submitted to AC '22)}\pause
	\begin{enumerate}[$\Rightarrow$]
		\item {\small 2. Range Arguments \cite{NguyenTWZ19}}\pause
		\item {\small 3. ZKAoK for Committed Symmetric Boolean Functions \cite{LingNPTW21}}\pause
		\item {\small 4a. Fully Dynamic Group Signatures (ext. \cite{NguyenTWZ19})}\pause
		\item {\small 4b. Repudiable-and-Claimable Ring Signatures (ext. \cite{NguyenTWZ19})}\pause
		\item {\small 5. Fully Dynamic Attribute-Based Signatures (submitted to AC '22)}
	\end{enumerate}
\end{frame}