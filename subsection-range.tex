% !TeX root = main.tex
\begin{frame}
	\underline{Contribution 2.} Code-Based Range Arguments for Signed Fractional Numbers
	
	{\small with Khoa Nguyen, Huaxiong Wang, and Neng Zeng}
	
	in ASIACRYPT 2019 \cite{NguyenTWZ19}.
\end{frame}

\begin{frame}{2. Range Arguments for Signed Fractional Numbers}
	\underline{Motivation.} \pause
	\begin{itemize}
		\item Existing efficient constructions of range proofs and arguments consider non-negative integers.\pause
		\item Possible contexts: health data, financial data, academic records where information is sensitive.\pause
	\end{itemize}

	\underline{Result.} Code-based Stern-like ZKAoK for inequality between committed signed fractional numbers:\pause
	\begin{itemize}
		\item Given $\Com(X, \mathbf{r}_x), \Com(Y, \mathbf{r}_y)$, prove $X \leq Y$ (or $X < Y$).\pause 
		\item Signed and fractional.\pause
		\item Linear size communication cost.
	\end{itemize}
	
\end{frame}

\begin{frame}{2. Range Arguments for Signed Fractional Numbers}
	\underline{Technique.} For proving $X \leq Y$ in ZK, find $Z$ such that $X + Z = Y$.\pause
	
	Represent signed fractional numbers $X$ and $Z$ in two's complement:\pause
	\begin{align*}
		\sbin(X) = x_\ell \dots x_0\bullet x_{-1}\dots x_{-f} \text{ where } X = -x_\ell \cdot 2^{\ell} + \sum_{i = -f}^{\ell-1} x_i\cdot 2^i\\ 
		\sbin(Z) = z_\ell \dots z_0\bullet z_{-1}\dots z_{-f} \text{ where } Z = -z_\ell \cdot 2^{\ell} + \sum_{i = -f}^{\ell-1} z_i\cdot 2^i.
	\end{align*}\pause

	Apply a similar technique from Libert et al. \cite{LibertLNW18} with some modifications to prove $X + Z = Y$.
\end{frame}