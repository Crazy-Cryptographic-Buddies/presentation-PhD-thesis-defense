% !TeX root = main.tex
\begin{frame}
	\underline{Contribution 5.} Code-Based Fully Dynamic Attribute-Based Signature Scheme
	
	{\small with San Ling, Khoa Nguyen, Duong Hieu Phan, Huaxiong Wang and Yanhong Xu}
	
	in submission to ASIACRYPT 2022.
\end{frame}

\begin{frame}{5. Fully Dynamic Attribute-Based Signature Scheme}
	Attribute-based signature (ABS) scheme allows a signer having secret attribute $x$\pause
	\begin{itemize}
		\item to sign anonymously with associated public policy $P$.\pause
		\item $P(x) = 1$.\pause
	\end{itemize}
	
	\underline{Motivation.}\pause
	\begin{itemize}
		\item Expressiveness of policies.\pause
		\item Unrevocability of signing keys.\pause
		\item Diversity of computational assumptions.
	\end{itemize}
\end{frame}

\begin{frame}{5. Fully Dynamic Attribute-Based Signature Scheme}
	\underline{Results.} \pause
	
	New definition of fully dynamic attribute-based signatures.\pause
	
	Code-based fully dynamic attribute-based signature scheme:\pause
	\begin{itemize}
		\item Policy: circuits with NAND gates only.\pause
		\item Revocability of signing keys $\Rightarrow$ fully dynamic.\pause
		\item Regular Null Syndrome Decoding $\Rightarrow$ code-based assumptions.\pause
		\item Signature size logarithmic in the maximum number of issued attributes.
	\end{itemize}
\end{frame}

\begin{frame}{5. Fully Dynamic Attribute-Based Signature Scheme}
	\underline{Technique.}\pause
	\begin{enumerate}
		\item Defining fully dynamic attribute-based signatures:\pause
		\begin{itemize}
			\item Adapt definitional framework from Bootle et al. \cite{BootleCCGG16}.\pause
			\item Lifetime of system is divided into time epochs.\pause
		\end{itemize}
		\item Code-based fully dynamic attribute-based signature scheme:\pause
		\begin{itemize}
			\item Policies are represented by Boolean circuits.\pause
			\item User's attribute is $\mathbf{x}$. \pause
			\item Put $\mathbf{d} = \Com(\mathbf{x}, \mathbf{r})$ at the leaves of Merkle tree. \pause
			\item To sign, construct ZKAoK for (i) $\mathbf{d}$ is a leaf of the tree, (ii) $\mathbf{d}$ is commitment to $\mathbf{x}$ and (iii) $P(\mathbf{x}) = 1$.
		\end{itemize}
	\end{enumerate}
\end{frame}