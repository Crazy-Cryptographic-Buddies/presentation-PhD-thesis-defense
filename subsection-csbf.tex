% !TeX root = main.tex
\begin{frame}
	\underline{Contribution 3.} Code-Based Arguments for Committed Symmetric Boolean Functions
	
	{\small with San Ling, Khoa Nguyen, Duong Hieu Phan, and Huaxiong Wang}
	
	in PQCrypto 2021 \cite{LingNPTW21}.
\end{frame}

\begin{frame}{3. Arguments for Committed Symmetric Boolean Functions}
	\underline{Motivation.} \pause
	\begin{itemize}
		\item Zero-knowledge proofs/arguments exist for all NP statements consider public statements.\pause
		\item Public function $f$ and public output $b$, \pause
		
		show that $b = f(x)$ where $x$ is secret.\pause
		\item Example: show in ZK knowledge of $\mathbf{x}$ s.t. $\textsf{SHA-256}(\mathbf{x})$ is equal to\pause
		{\footnotesize \begin{equation*}
			e4a41c66866bd166fbc0f685dbd7fe28b7bc1e62c018067ef8e203a35eaf210b
		\end{equation*}}\pause
		\item Contexts where $f$ is private are not much investigated.
	\end{itemize}
\end{frame} 
\begin{frame}{3. Arguments for Committed Symmetric Boolean Functions}
	\underline{Result.}\pause
	\begin{itemize}
		\item ZKAoK for correct evaluations of committed functions on committed inputs.\pause
		\item Committed functions a symmetric Boolean functions.\pause
		\item Symmetric Boolean function (SBF) $f : \{0,1\}^n \to \{0,1\}$ represented by \pause
		\begin{equation*}
			\mathbf{v}(f) = (v_0, \dots, v_n).\pause
		\end{equation*}
		On input $\mathbf{x} \in \{0,1\}^n$, returns $f(\mathbf{x}) = v_w$ where $w = \weight(\mathbf{x})$.\pause
		\item Complexity linear in input's size.
	\end{itemize}
\end{frame}

\begin{frame}{3. Arguments for Committed Symmetric Boolean Functions}
	\underline{Technique.}\pause
	\begin{enumerate}
		\item Construct $\mathbf{y} = (y_0, \dots, y_n)$ the $w$-th basis vector.\pause
		\item Use equivalence $f(\mathbf{x}) = b \iff v_w = b \iff \langle\mathbf{v}, \mathbf{y}\rangle=b$.\pause
		\begin{itemize}
			\item Compute $\mathbf{c} := \extCsbf(\mathbf{y}) \oplus \encCsbf(\mathbf{v}) = (\mathbf{c}_0 \Vert \mathbf{c}_1\Vert \dots\Vert \mathbf{c}_n)$.\pause
		\end{itemize}
		\item Observe\pause
		\begin{itemize}
			\item Block $\mathbf{c}_w$ has $2$ same bits equal to $b$.\pause
			\item For $i \neq w$, $\weight(\mathbf{c}_i)$ is odd.
		\end{itemize}
	\end{enumerate}
\end{frame}